\documentclass[12pt]{article}

\usepackage[utf8]{inputenc}
\usepackage[english, russian]{babel}

% "А вот теперь --- слайды!"
\usepackage{slides}

\IfFileExists{cyrtimes.sty}
    {
        \usepackage{cyrtimespatched}
    }
    {
        % А если Times нету, то будет CM...
    }
\usepackage{graphicx}

\def\Person{Шапошников А.А.}
\def\Title{Тесты в frontik common}
\def\FooterTitle{\Title}
\def\SubTitle{Пара идей о том, как можно тестировать приложения frontik}
\newcommand{\TitleSlide}{
    \addcontentsline{toc}{section}{\Title}%
    ~\vspace{1cm}

    \begin{center}
    {\huge \begin{spacing}{1}\Title\end{spacing}}

    {\SubTitle}
    \vspace{2cm}

    \ifthenelse{\isundefined{\Student}}{}
        {\small Студент: \Student\\}
    \ifthenelse{\isundefined{\Advisor}}{}
        {\small Руководитель: \Advisor\\}
    \ifthenelse{\isundefined{\Person}}{}
        {\Person\\}
    \ifthenelse{\isundefined{\Affilation}}{}
        {\Affilation\\}
    \end{center}
    \thispagestyle{empty}
}


%% Переносы в презентации смотряся не очень.
\hyphenpenalty 10000
\sloppy


% pygments какие-то стили
\usepackage{fancyvrb}
\usepackage{color}
%%%%%%%%%%%%%%%%%%%%%%%%%%%\usepackage[ascii]{inputenc}

\makeatletter
\def\PY@reset{\let\PY@it=\relax \let\PY@bf=\relax%
    \let\PY@ul=\relax \let\PY@tc=\relax%
    \let\PY@bc=\relax \let\PY@ff=\relax}
\def\PY@tok#1{\csname PY@tok@#1\endcsname}
\def\PY@toks#1+{\ifx\relax#1\empty\else%
    \PY@tok{#1}\expandafter\PY@toks\fi}
\def\PY@do#1{\PY@bc{\PY@tc{\PY@ul{%
    \PY@it{\PY@bf{\PY@ff{#1}}}}}}}
\def\PY#1#2{\PY@reset\PY@toks#1+\relax+\PY@do{#2}}

\def\PY@tok@gd{\def\PY@tc##1{\textcolor[rgb]{0.63,0.00,0.00}{##1}}}
\def\PY@tok@gu{\let\PY@bf=\textbf\def\PY@tc##1{\textcolor[rgb]{0.50,0.00,0.50}{##1}}}
\def\PY@tok@gt{\def\PY@tc##1{\textcolor[rgb]{0.00,0.25,0.82}{##1}}}
\def\PY@tok@gs{\let\PY@bf=\textbf}
\def\PY@tok@gr{\def\PY@tc##1{\textcolor[rgb]{1.00,0.00,0.00}{##1}}}
\def\PY@tok@cm{\let\PY@it=\textit\def\PY@tc##1{\textcolor[rgb]{0.25,0.50,0.50}{##1}}}
\def\PY@tok@vg{\def\PY@tc##1{\textcolor[rgb]{0.10,0.09,0.49}{##1}}}
\def\PY@tok@m{\def\PY@tc##1{\textcolor[rgb]{0.40,0.40,0.40}{##1}}}
\def\PY@tok@mh{\def\PY@tc##1{\textcolor[rgb]{0.40,0.40,0.40}{##1}}}
\def\PY@tok@go{\def\PY@tc##1{\textcolor[rgb]{0.50,0.50,0.50}{##1}}}
\def\PY@tok@ge{\let\PY@it=\textit}
\def\PY@tok@vc{\def\PY@tc##1{\textcolor[rgb]{0.10,0.09,0.49}{##1}}}
\def\PY@tok@il{\def\PY@tc##1{\textcolor[rgb]{0.40,0.40,0.40}{##1}}}
\def\PY@tok@cs{\let\PY@it=\textit\def\PY@tc##1{\textcolor[rgb]{0.25,0.50,0.50}{##1}}}
\def\PY@tok@cp{\def\PY@tc##1{\textcolor[rgb]{0.74,0.48,0.00}{##1}}}
\def\PY@tok@gi{\def\PY@tc##1{\textcolor[rgb]{0.00,0.63,0.00}{##1}}}
\def\PY@tok@gh{\let\PY@bf=\textbf\def\PY@tc##1{\textcolor[rgb]{0.00,0.00,0.50}{##1}}}
\def\PY@tok@ni{\let\PY@bf=\textbf\def\PY@tc##1{\textcolor[rgb]{0.60,0.60,0.60}{##1}}}
\def\PY@tok@nl{\def\PY@tc##1{\textcolor[rgb]{0.63,0.63,0.00}{##1}}}
\def\PY@tok@nn{\let\PY@bf=\textbf\def\PY@tc##1{\textcolor[rgb]{0.00,0.00,1.00}{##1}}}
\def\PY@tok@no{\def\PY@tc##1{\textcolor[rgb]{0.53,0.00,0.00}{##1}}}
\def\PY@tok@na{\def\PY@tc##1{\textcolor[rgb]{0.49,0.56,0.16}{##1}}}
\def\PY@tok@nb{\def\PY@tc##1{\textcolor[rgb]{0.00,0.50,0.00}{##1}}}
\def\PY@tok@nc{\let\PY@bf=\textbf\def\PY@tc##1{\textcolor[rgb]{0.00,0.00,1.00}{##1}}}
\def\PY@tok@nd{\def\PY@tc##1{\textcolor[rgb]{0.67,0.13,1.00}{##1}}}
\def\PY@tok@ne{\let\PY@bf=\textbf\def\PY@tc##1{\textcolor[rgb]{0.82,0.25,0.23}{##1}}}
\def\PY@tok@nf{\def\PY@tc##1{\textcolor[rgb]{0.00,0.00,1.00}{##1}}}
\def\PY@tok@si{\let\PY@bf=\textbf\def\PY@tc##1{\textcolor[rgb]{0.73,0.40,0.53}{##1}}}
\def\PY@tok@s2{\def\PY@tc##1{\textcolor[rgb]{0.73,0.13,0.13}{##1}}}
\def\PY@tok@vi{\def\PY@tc##1{\textcolor[rgb]{0.10,0.09,0.49}{##1}}}
\def\PY@tok@nt{\let\PY@bf=\textbf\def\PY@tc##1{\textcolor[rgb]{0.00,0.50,0.00}{##1}}}
\def\PY@tok@nv{\def\PY@tc##1{\textcolor[rgb]{0.10,0.09,0.49}{##1}}}
\def\PY@tok@s1{\def\PY@tc##1{\textcolor[rgb]{0.73,0.13,0.13}{##1}}}
\def\PY@tok@sh{\def\PY@tc##1{\textcolor[rgb]{0.73,0.13,0.13}{##1}}}
\def\PY@tok@sc{\def\PY@tc##1{\textcolor[rgb]{0.73,0.13,0.13}{##1}}}
\def\PY@tok@sx{\def\PY@tc##1{\textcolor[rgb]{0.00,0.50,0.00}{##1}}}
\def\PY@tok@bp{\def\PY@tc##1{\textcolor[rgb]{0.00,0.50,0.00}{##1}}}
\def\PY@tok@c1{\let\PY@it=\textit\def\PY@tc##1{\textcolor[rgb]{0.25,0.50,0.50}{##1}}}
\def\PY@tok@kc{\let\PY@bf=\textbf\def\PY@tc##1{\textcolor[rgb]{0.00,0.50,0.00}{##1}}}
\def\PY@tok@c{\let\PY@it=\textit\def\PY@tc##1{\textcolor[rgb]{0.25,0.50,0.50}{##1}}}
\def\PY@tok@mf{\def\PY@tc##1{\textcolor[rgb]{0.40,0.40,0.40}{##1}}}
\def\PY@tok@err{\def\PY@bc##1{\fcolorbox[rgb]{1.00,0.00,0.00}{1,1,1}{##1}}}
\def\PY@tok@kd{\let\PY@bf=\textbf\def\PY@tc##1{\textcolor[rgb]{0.00,0.50,0.00}{##1}}}
\def\PY@tok@ss{\def\PY@tc##1{\textcolor[rgb]{0.10,0.09,0.49}{##1}}}
\def\PY@tok@sr{\def\PY@tc##1{\textcolor[rgb]{0.73,0.40,0.53}{##1}}}
\def\PY@tok@mo{\def\PY@tc##1{\textcolor[rgb]{0.40,0.40,0.40}{##1}}}
\def\PY@tok@kn{\let\PY@bf=\textbf\def\PY@tc##1{\textcolor[rgb]{0.00,0.50,0.00}{##1}}}
\def\PY@tok@mi{\def\PY@tc##1{\textcolor[rgb]{0.40,0.40,0.40}{##1}}}
\def\PY@tok@gp{\let\PY@bf=\textbf\def\PY@tc##1{\textcolor[rgb]{0.00,0.00,0.50}{##1}}}
\def\PY@tok@o{\def\PY@tc##1{\textcolor[rgb]{0.40,0.40,0.40}{##1}}}
\def\PY@tok@kr{\let\PY@bf=\textbf\def\PY@tc##1{\textcolor[rgb]{0.00,0.50,0.00}{##1}}}
\def\PY@tok@s{\def\PY@tc##1{\textcolor[rgb]{0.73,0.13,0.13}{##1}}}
\def\PY@tok@kp{\def\PY@tc##1{\textcolor[rgb]{0.00,0.50,0.00}{##1}}}
\def\PY@tok@w{\def\PY@tc##1{\textcolor[rgb]{0.73,0.73,0.73}{##1}}}
\def\PY@tok@kt{\def\PY@tc##1{\textcolor[rgb]{0.69,0.00,0.25}{##1}}}
\def\PY@tok@ow{\let\PY@bf=\textbf\def\PY@tc##1{\textcolor[rgb]{0.67,0.13,1.00}{##1}}}
\def\PY@tok@sb{\def\PY@tc##1{\textcolor[rgb]{0.73,0.13,0.13}{##1}}}
\def\PY@tok@k{\let\PY@bf=\textbf\def\PY@tc##1{\textcolor[rgb]{0.00,0.50,0.00}{##1}}}
\def\PY@tok@se{\let\PY@bf=\textbf\def\PY@tc##1{\textcolor[rgb]{0.73,0.40,0.13}{##1}}}
\def\PY@tok@sd{\let\PY@it=\textit\def\PY@tc##1{\textcolor[rgb]{0.73,0.13,0.13}{##1}}}

\def\PYZbs{\char`\\}
\def\PYZus{\char`\_}
\def\PYZob{\char`\{}
\def\PYZcb{\char`\}}
\def\PYZca{\char`\^}
% for compatibility with earlier versions
\def\PYZat{@}
\def\PYZlb{[}
\def\PYZrb{]}
\makeatother
% /pygments

\begin{document}


\TitleSlide

\section{Цель}

\emph{Целью} является обеспечить test first разработку во фронтике.

+ покрытие тестами его кода для регрессии

+ ещё больше развязать разработку и верстку


\section{Требования к тестам}

\begin{enumerate}
\item должны быть быстрыми
\item не должны зависеть от способа их запуска и чего-то требовать от окружения (т.е. unit тесты)
\item должны писаться сравнительно легко
\item сама система их поддержки должна быть достаточно гибкой и быть написана достаточно быстро
\end{enumerate}

\section{А чего, напротив, не требуется}

\begin{enumerate}
\item "строгих" проверок
\item проверок на гонки
\item 100\verb+%+ эмуляции HTTP
\end{enumerate}

\section{Моки сервисов хотелось бы писать вот так}
\small
\begin{Verbatim}[commandchars=\\\{\}]
\PY{k}{class} \PY{n+nc}{SofeaEmployerMock}\PY{p}{(}\PY{n}{ServiceMock}\PY{p}{)}\PY{p}{:}
 \PY{k}{def} \PY{n+nf}{fetch\PYZus{}request}\PY{p}{(}\PY{n+nb+bp}{self}\PY{p}{,} \PY{n}{req}\PY{p}{,} \PY{n}{callback} \PY{o}{=} \PY{n+nb+bp}{None}\PY{p}{)}\PY{p}{:}
   \PY{k}{if} \PY{l+s}{'}\PY{l+s}{/vacancy/short?lang=RU&id=1}\PY{l+s}{'} \PY{o}{==} \PY{n}{req}\PY{o}{.}\PY{n}{url}\PY{p}{:}
     \PY{k}{return} \PY{l+m+mi}{200}\PY{p}{,} 
       \PY{l+s+sd}{'''<?xml version="1.0" encoding="UTF-8" standalone="yes"?>}
\PY{l+s+sd}{           <vacancies>}
\PY{l+s+sd}{             <vacancy>}
\PY{l+s+sd}{               <vacancyId>1</vacancyId>}
\PY{l+s+sd}{               <name>Merchandizer</name>}
\PY{l+s+sd}{               <company>}
\PY{l+s+sd}{                 <id>63357</id>}
\PY{l+s+sd}{                 <name>PROFPARK, ZAO</name>}
\PY{l+s+sd}{             </vacancy>}
\PY{l+s+sd}{           </vacancies>'''}
\end{Verbatim}

\section{Вот так мог бы выглядеть тест}

\small
\begin{Verbatim}[commandchars=\\\{\}]
\PY{n+nb+bp}{self}\PY{o}{.}\PY{n}{register}\PY{p}{(}\PY{n}{SofeaEmployerMock}\PY{p}{(}\PY{p}{)}\PY{p}{)}
\PY{n+nb+bp}{self}\PY{o}{.}\PY{n}{execute\PYZus{}async\PYZus{}method}\PY{p}{(}
    \PY{n+nb+bp}{self}\PY{o}{.}\PY{n}{handler}\PY{p}{,} \PY{n}{vacancy\PYZus{}list\PYZus{}page\PYZus{}by\PYZus{}manager}\PY{p}{,} \PY{l+s}{'}\PY{l+s}{3}\PY{l+s}{'}\PY{p}{,} \PY{l+s}{'}\PY{l+s}{1}\PY{l+s}{'}\PY{p}{)}
\PY{n}{doc} \PY{o}{=} \PY{n+nb+bp}{self}\PY{o}{.}\PY{n}{handler}\PY{o}{.}\PY{n}{doc}\PY{o}{.}\PY{n}{root\PYZus{}node}
\PY{n+nb+bp}{self}\PY{o}{.}\PY{n}{assertEqual}\PY{p}{(}
    \PY{n}{doc}\PY{o}{.}\PY{n}{xpath}\PY{p}{(}
        \PY{l+s}{'}\PY{l+s}{//employer/employerManager/fullName/text()}\PY{l+s}{'}\PY{p}{)}\PY{p}{[}\PY{l+m+mi}{0}\PY{p}{]}\PY{p}{,}
                 \PY{l+s}{u'}\PY{l+s}{Evreni Petrovich}\PY{l+s}{'}\PY{p}{)}
\PY{n+nb+bp}{self}\PY{o}{.}\PY{n}{assertEqual}\PY{p}{(}
    \PY{n}{doc}\PY{o}{.}\PY{n}{xpath}\PY{p}{(}
        \PY{l+s}{'}\PY{l+s}{//employer/@id}\PY{l+s}{'}\PY{p}{)}\PY{p}{[}\PY{l+m+mi}{0}\PY{p}{]}\PY{p}{,} \PY{l+s}{'}\PY{l+s}{1}\PY{l+s}{'}\PY{p}{)}
\PY{n+nb+bp}{self}\PY{o}{.}\PY{n}{assertEqual}\PY{p}{(}
    \PY{n}{doc}\PY{o}{.}\PY{n}{xpath}\PY{p}{(}
        \PY{l+s}{'}\PY{l+s}{//employer/russia/text()}\PY{l+s}{'}\PY{p}{)}\PY{p}{[}\PY{l+m+mi}{0}\PY{p}{]}\PY{p}{,} \PY{l+s}{'}\PY{l+s}{true}\PY{l+s}{'}\PY{p}{)}
\end{Verbatim}

\verb+vacancy_list_page_by_manager+ - это имя тестируемого метода

\section{Куда можно двигаться дальше}

\begin{enumerate}
\item Самое первое - это отказаться от наивного вручную написанного xml в моках внешних сервисов, не потеряв при этом простоты и наглядности
\item Проблемы поменьше:
\begin{itemize}
\item поддержка \emph{POST}, \emph{DELETE} и \emph{PUT}
\item поддержка Content-type \emph{не} XML

\end{itemize}

\end{enumerate}

\end{document}

