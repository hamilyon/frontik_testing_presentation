\documentclass[12pt]{article}

\usepackage[utf8]{inputenc}
\usepackage[english, russian]{babel}
\usepackage{minted}

% "А вот теперь --- слайды!"
\usepackage{slides}

\usepackage{graphicx}
\usepackage{pdfpages}

\def\Person{Шапошников А.А.}
\def\Title{Тесты во фронтике}
\def\FooterTitle{\Title}
\def\SubTitle{Пара идей о том, как можно тестировать приложения frontik}
\input{title.tex}

%% Переносы в презентации смотряся не очень.
\hyphenpenalty 10000
\sloppy


% pygments какие-то стили
% /pygments

\usepackage{fontspec} % loaded by polyglossia, but included here for transparency
\usepackage{polyglossia}
\setmainlanguage{russian}
\setotherlanguage{english}

% XeLaTeX can use any font installed in your system fonts folder
% Linux Libertine in the next line can be replaced with any
% OpenType or TrueType font that supports the Cyrillic script.
\setromanfont[Mapping=tex-text]{Liberation Serif}
\setsansfont[Mapping=tex-text]{Liberation Sans} % Without this Cyrillic Sans Serif text won't show.
\setmonofont[Mapping=tex-text]{Nimbus Mono L}


\begin{document}

\includegraphics{logo.png}
\TitleSlide

\section{Мотивация и цель}
"TDD во фронтике"

Обеспечить слой logic системой, позволяющей отлавливать баги связанные с изменением контрактов с внешними сервисами

В целом дружить с нашей экосистемой

\section{Требования к тестам}

\begin{enumerate}
\item Наглядность и читаемость, совместимость с unittest инфраструктурой python
\item Не должны зависеть от способа их запуска и чего-то требовать от окружения (т.е. unit тесты)
\end{enumerate}

\section{}
\begin{figure}
\includegraphics[page=1, scale=1]{frontikarchitecture.pdf}
%\includepdf[page=1]{frontikarchitecture.pdf}
\end{figure}

\section{Тестовый фреймворк}

\begin{itemize}
\item Приложения frontik ходят в бэкэнды, затем опционально накладывают xsl
\item Такая узкая специализация позволяет встроиться и перехватывать запросы в сетевом стеке и приложение изолировано

\end{itemize}

\section{}
\begin{figure}
\includegraphics[page=1, scale=2]{interconnection.pdf}
%\includepdf[pages={1}]{}
\end{figure}

\section{Ключевые решения в ходе разработки}

\begin{itemize}
\item Встроен во frontik. Использует настоящий экземпляр frontik c подмененным http-client
\item \emph{Нет} tornado/ioloop, callback'и вызываются через тестовый фреймворк
\item \emph{Нет} HTTP и сокетов, http-client напрямую ходит к мокам сервисов
\item собственно наложение xsl можно тестировать, но зачем?

\end{itemize}

\section{Стандартный подход к тестированию}
\begin{itemize}
\item setUp подготавливает общую часть

\item caveat: не поддерживается более одного вызова call в одном тесте
\end{itemize}

\section{Код под тестом}
\small

\include{test2}
вставьте здесь

\section{Тест}

\small
\inputminted[linenos=true]{python}{test.py}

\verb+vacancy_list_page_by_manager+ - это метод под тестом

\verb+self.register+, \verb+self.handler+, \verb+execute_async_method+ - из суперкласса

\section{Вызов тестирования интеграционного слоя}

Почти все тесты ходят за одними и теми же данными (i.e. вакансия, сессия), и в то же время очевидно не всем подойдет один и тот же стандартный ответ

Ответы генерируют внешние сервисы, которые лежат в другом репозитории и обновляются в случайный момент времени

Если контракт одного HTTP ресурса поменлся, тесты (в идеальном мире) показывают, что код, существенно его использующий, сломался

Как ориентироваться в моках?

\section{Репозиторий моков}

Напротив, развязан c frontik'ом и даже с питоном

Как оказалось, одного xml мало, мы храним рядом с ним ещё и необходимые header'ы

И ещё отдельно пару сервис-ресурс

\section{Куда можно двигаться дальше}

\begin{enumerate}
\item Документация
\item Более внятные сообщения об ошибках
\end{enumerate}

\section{Предлагаю критиковать}



\end{document}

